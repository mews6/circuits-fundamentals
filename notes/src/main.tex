%%%%%%%%%%%%%%%%%%%%%%%%%%%%%%%%%%%%%%%%%
% Important note:
% Chapter heading images should have a 2:1 width:height ratio,
% e.g. 920px width and 460px height.
%
% The original template (the Legrand Orange Book Template) can be found here --> http://www.latextemplates.com/template/the-legrand-orange-book
%
% Original author of the Legrand Orange Book Template:
% Mathias Legrand (legrand.mathias@gmail.com) with modifications by:
% Vel (vel@latextemplates.com)
%
% Original License:
% CC BY-NC-SA 3.0 (http://creativecommons.org/licenses/by-nc-sa/3.0/)
%%%%%%%%%%%%%%%%%%%%%%%%%%%%%%%%%%%%%%%%%
 
%----------------------------------------------------------------------------------------
%	PACKAGES AND OTHER DOCUMENT CONFIGURATIONS
%----------------------------------------------------------------------------------------

\documentclass[11pt,fleqn]{book} % Default font size and left-justified equations

\usepackage[top=3cm,bottom=3cm,left=3.2cm,right=3.2cm,headsep=10pt,letterpaper]{geometry} % Page margins

\usepackage{xcolor} % Required for specifying colors by name
\definecolor{ocre}{RGB}{52,177,201} % Define the orange color used for highlighting throughout the book

% Font Settings
\usepackage{avant} % Use the Avantgarde font for headings
%\usepackage{times} % Use the Times font for headings
\usepackage{mathptmx} % Use the Adobe Times Roman as the default text font together with math symbols from the Sym­bol, Chancery and Com­puter Modern fonts

\usepackage{microtype} % Slightly tweak font spacing for aesthetics
\usepackage[utf8]{inputenc} % Required for including letters with accents
\usepackage[T1]{fontenc} % Use 8-bit encoding that has 256 glyphs

% Bibliography
\usepackage[style=alphabetic,sorting=nyt,sortcites=true,autopunct=true,babel=hyphen,hyperref=true,abbreviate=false,backref=true,backend=biber]{biblatex}
\addbibresource{bibliography.bib} % BibTeX bibliography file
\defbibheading{bibempty}{}

\input{structure} % Insert the commands.tex file which contains the majority of the structure behind the template

\begin{document}
\title{Clustering the interstellar medium}

%----------------------------------------------------------------------------------------
%	TITLE PAGE
%----------------------------------------------------------------------------------------

\begingroup
\thispagestyle{empty}
\AddToShipoutPicture*{\put(0,0){\includegraphics[scale=1.25]{esahubble}}} % Image background
\centering
\vspace*{5cm}
\par\normalfont\fontsize{35}{35}\sffamily\selectfont
\textbf{Circuits Fundamentals}\\
{\LARGE github.com/mews6}\par % Book title
\vspace*{1cm}
{Jaime Torres}\par % Author name
\endgroup

%----------------------------------------------------------------------------------------
%	COPYRIGHT PAGE
%----------------------------------------------------------------------------------------

\newpage
~\vfill
\thispagestyle{empty}

%\noindent Copyright \copyright 
\noindent \textit{First release, 2024} % Printing/edition date

%----------------------------------------------------------------------------------------
%	TABLE OF CONTENTS
%----------------------------------------------------------------------------------------

\chapterimage{head1.png} % Table of contents heading image

\pagestyle{empty} % No headers

\tableofcontents % Print the table of contents itself

%\cleardoublepage % Forces the first chapter to start on an odd page so it's on the right

\pagestyle{fancy} % Print headers again

%----------------------------------------------------------------------------------------
%	CHAPTER 1
%----------------------------------------------------------------------------------------

\chapterimage{head2.png} % Chapter heading image

\chapter{Introduction}

\chapter{Electricity's Fundamentals}

Electricity is, in simple terms, (...)

Positive charges and negative charges are, per charge:

\begin{gather}
    Positive: \ 1.6x10^{-19}C \\
    Negative: \ 6.24x10^{18}e
\end{gather}

They're both in the measuring unit of the other.

\section{Coulomb's Law:}

We mathematically define Coulomb's Law as:

\begin{gather}
    F = K \cdot (\frac{Q\cdot q}{r^2})
\end{gather}

a few important constants come from:

\begin{gather}
    K = \frac{1}{4\pi\cdot \epsilon}\\
\end{gather}

\section{Current}

When we have a group of charging moving at the same time, we can measure it in amperes, the specific 
measure can be explained mathematically as:

\begin{gather}
    [I] = A = \frac{1C}{1S}
\end{gather}

This is a measure of unit that has a direction, and is inherently vectorial.
It is also the main way we'll measure electricity in this course

\section[short]{Resistivity}

It is possible to control Current through resistant materials that can stop partially the flow of electrons. 
This comes in a sense, as a consequence of Ohm's Law. 

We can calculate resistivity as:

\begin{gather}
    R = \rho(T^o) x (\frac{L}{A})
\end{gather}

This 'R' we have in the previous formula is actually an Ohm, and can also be written as $\Omega$.

\subsection{Effects driven by temperature}

Temperature can make it so there are more or fewer free electrons. The higher the temperature, the more free electrons;
In a sense this happens because heat physically is a measure of how much (...). The fact there are little to no electrons helps 
the current technology on superconductivity.   

\subsection{Aislants}

An Aislant is meant to make resistivity as large as possible, and even though perfect aislants are, in practice, impossible to achieve,
it makes the flow of electrons negligible. 

\section{Conductivity}

The opposite of Resistivity is conductivity. This is measured in Siemens, and can be expressed as:

\begin{gather}
    [G] = S = \frac{1}{\Omega}
\end{gather}

\section{Closed and Open Circuits}

We can think on circuits being broadly classified in two: An open circuit, which implies (...), or a closed circuit, which
implies (...).

\chapter{Kirchoff's Laws}



\end{document}